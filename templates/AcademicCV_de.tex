% !TEX program = xelatex
\documentclass[
    fontsize=11pt,
    a4paper,
]{scrartcl}

\usepackage{typearea}
\areaset{16cm}{25cm}

\usepackage{array}
\usepackage{tabularx}

\usepackage[usenames,dvipsnames]{xcolor}
\usepackage{hyperref}
\definecolor{linkcolour}{rgb}{0,0.2,0.6}
\hypersetup{
    colorlinks,
    breaklinks,
    urlcolor=black,
    linkcolor=black
}

\usepackage{fontawesome5}

\usepackage{biblatex}
\addbibresource{cv-data/publications.bib}
\nocite{*}

\usepackage{titlesec}
\titleformat{\section}{\large\bfseries\raggedright}{}{0em}{}[\titlerule]
\titlespacing{\section}{0pt}{10pt}{4pt}

\usepackage[useregional=numeric]{datetime2}
\DTMusemodule{english}{en-GB}
\DTMsetregional
\DTMlangsetup{showdayofmonth=false}

\newcommand{\VAR}[1]{} % Empty variable for jinja templating
\newcommand{\BLOCK}[1]{} % Empty variable for jinja templating

\begin{document}
\pagestyle{empty}

% Name and contact information
\BLOCK{if general}\BLOCK{set data = general['de']}
\begin{center}
    {
        \large \textbf{Curriculum Vitae - \VAR{data.name}} \href{https://\VAR{data.other_links.Orcid}}{\faOrcid}
    }
    \vspace{8pt}

    \begin{tabular}[t]{ c c c }
        \href{mailto:\VAR{data.email}}{\VAR{data.email}} & \href{https://\VAR{data.website}}{\VAR{data.website}} & \href{https://\VAR{data.other_links.Orcid}}{\VAR{data.other_links.Orcid}} \\
    \end{tabular}
\end{center}
\BLOCK{endif}

%
% Education
%
\BLOCK{if education}\BLOCK{set data = education['de']}

\section{Ausbildung}
\noindent
\begin{tabularx}{\textwidth}{@{} m{8em} X}
\BLOCK{ if data.current is not none }
    \BLOCK{for station in data.current}
        Seit \textsc{\DTMdate{\VAR{data.current[station].date}}} & \textbf{\VAR{data.current[station].name}}, \VAR{data.current[station].institution}, \VAR{data.current[station].location} (erwartet: \VAR{data.current[station].grade}) \\
        \BLOCK{if data.current[station].notes}
            & \VAR{data.current[station].notes} \\
        \BLOCK{ endif }
    \BLOCK{ endfor }
\BLOCK{ endif }

\BLOCK{if data.former is not none}
    \BLOCK{for station in data.former}
        \DTMdate{\VAR{data.former[station].date}} & \textbf{\VAR{data.former[station].name}}, \VAR{data.former[station].institution}, \VAR{data.former[station].location} (\VAR{data.former[station].grade}) \\
        \BLOCK{if data.former[station].notes}
            & \VAR{data.former[station].notes} \\
        \BLOCK{ endif }
    \BLOCK{endfor}
\BLOCK{endif}
\end{tabularx}
\BLOCK{endif}

%
% Work experience
%
\BLOCK{if work_experience}\BLOCK{set data = work_experience['de']}

\section{Berufserfahrung}
\noindent
\begin{tabularx}{\textwidth}{@{} m{8em} X}
\BLOCK{ if data.current is not none }
    \BLOCK{ for station in data.current }
        Seit \textsc{\DTMdate{\VAR{data.current[station].startdate}}} & \textbf{\VAR{data.current[station].name}}, \VAR{data.current[station].institution}, \VAR{ data.current[station].location } \\
    \BLOCK{ endfor }
\BLOCK{ endif }

\BLOCK{ if data.current is not none }
    \BLOCK{for station in data.former}
        \DTMdate{\VAR{data.former[station].startdate}} - \DTMdate{\VAR{data.former[station].enddate}} & \VAR{data.former[station].position}, \VAR{data.former[station].company}, \VAR{data.former[station].location} \\
    \BLOCK{ endfor }
\BLOCK{ endif }
\end{tabularx}
\BLOCK{endif}

%
% Publications
%
%\section{Publications}

%\printbibliography[heading=none]

%
% Schools and conferences
%
\BLOCK{if schools_conferences}\BLOCK{set data = schools_conferences['de']}

\section{Schulen und Konferenzen}
\noindent
\begin{tabularx}{\textwidth}{@{} m{8em} X}
\BLOCK{ for school in data }
\textsc{\DTMdate{\VAR{data[school].enddate}}} & \textbf{\VAR{data[school].name}}, \VAR{data[school].location} \\
\BLOCK{ if data[school].poster_title }
 & Poster: \VAR{data[school].poster_title}
\BLOCK{endif}
\BLOCK{ endfor }
\end{tabularx}
\BLOCK{endif}


%
% Technical Skills
%
\BLOCK{if technical_skills}\BLOCK{set data = technical_skills['de']}

\section{Technische Skills}
\noindent
\begin{tabularx}{\textwidth}{@{} X}
\BLOCK{ if data.languages is not none }
Sprachen: \BLOCK{ for language in data.languages }\VAR{ data.languages[language].name }\BLOCK{ if loop.last is false}, \BLOCK{ endif }\BLOCK{ endfor } \\
\BLOCK{ endif }

\BLOCK{ if data.technologies is not none }
Sonstige: \BLOCK{ for other in data.technologies }\VAR{ data.technologies[other].name }\BLOCK{ if loop.last is false}, \BLOCK{ endif }\BLOCK{ endfor } \\
\BLOCK{ endif }

\end{tabularx}
\BLOCK{endif}

\end{document}
